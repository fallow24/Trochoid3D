\section{Conclusions}

In this work, we have addressed the prediction of the trochoidal motion that a LiDAR sensor experiences when mounted rigidly inside a spherical mobile mapping system.
Our new calibration procedure, which is specifically designed for this use case, estimates the extrinsic LiDAR-to-center-point parameters needed to construct the trochoidal trajectory.
We evaluated the predicted motion model trajectory and compared it to the trajectories that state-of-the-art LIO methods produce.
The results show that the motion model trajectory better resembles the trochoidal geometry than the state-of-the-art approaches, which have no information on the spherical nature of the system.
Nevertheless, a lot of work remains to be done.
We need to measure the ground normal vector using the LiDAR data to account for rolling on slopes.
Additionally, since our motion model only utilizes IMU data it is still prone to drift, thus we do not plan on using it as is.
Furthermore, the state-of-the-art LIO approaches perform sub-optimally due to the motion profile of the rolling ball, which they were not designed for.
Thus, in future work, we plan on utilizing our motion model to implement a LIO method that is suited for this context.
Another solution is to modify existing LIO methods, e.g., bootstrap them with our motion model or modify their state propagation mechanisms.
Our results indicate that DLIO~\cite{10160508} is a promising approach for such modifications due to the overall better performance.

