\section{Related work}

In the previous section, we have already introduced some spherical systems and their possible applications.
Our paper focuses on two major topics: 1) Extrinsic LiDAR calibration and 2) LiDAR-inertial odometry (LIO).
Note that the motion model presented in the paper only falls into the category of \textit{Inertial} odometry, yet we compare it with other LIO methods.
 
\subsection{Extrinsic LiDAR calibration}

Finding the extrinsic parameters between the coordinate systems of a LiDAR and some other sensor or reference frame is a broad, well studied field.
State-of-the-art approaches do not require the user to place artificial external markers in the environment, but utilize the features or geometries of the environment directly.
The following methods address the calibration LiDAR-IMU systems.
In~\cite{Liu2020NovelMultifeature} the authors introduced an on-site calibration method for LiDAR-IMU systems that combines point, sphere, line, cylinder, and plane features.
Their approach employs a full information maximum likelihood estimation to obtain both the LiDAR-to-IMU extrinsics, but also the IMU intrinsic parameters.
Another similar appraoch that estimates both LiDAR-IMU extrinsic and IMU intrinsic parameters is~\cite{Liu2019ErrorModeling}, where the authors also utilize point, plane, cone, and cylinder features to construct a geometrically constrained optimization problem, followed by a restricted maximum likelihood estimation.


Li et al. (2021) present a high-accuracy autocalibration method for LiDAR and IMU systems in "3D LiDAR/IMU Calibration Based on Continuous-Time Trajectory Estimation in Structured Environments." Their approach utilizes Gaussian process regression to model the IMU trajectory, allowing for on-manifold batch optimization of distorted and delayed LiDAR points. This method is especially effective in known environments with structured planes, offering substantial improvements in calibration accuracy.

Lv et al. (2022) address the calibration of LiDAR-IMU systems with their work "Observability-Aware Intrinsic and Extrinsic Calibration of LiDAR-IMU Systems," introducing a continuous-time batch-optimization framework. This method enhances efficiency and provides accurate calibration results even under degenerate cases by leveraging observability-aware modules, including an information-theoretic data selection policy and a state update mechanism that updates only the identifiable directions in the state space.

\subsection{LiDAR-inertial odometry}